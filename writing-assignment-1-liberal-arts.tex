\documentclass[12pt]{article}
\usepackage[utf8]{inputenc}
\usepackage[T1]{fontenc}
%
% Margin - 1 inch on all sides
%
\usepackage[letterpaper]{geometry}
\usepackage{times}
\geometry{top=1.0in, bottom=1.0in, left=1.0in, right=1.0in}

%
% Double-spacing
%
\usepackage{setspace}
\doublespacing

%
% Rotating tables (e.g. sideways when too long)
%
\usepackage{rotating}

\usepackage{nth}


\usepackage{graphicx}
\usepackage{grffile}
\usepackage{longtable}
\usepackage{wrapfig}

%
% Fancy-header package to modify header/page numbering (insert last name)
%
\usepackage{fancyhdr}
\pagestyle{fancy}
\lhead{}
\chead{}
\rhead{Nelson \thepage}
\lfoot{}
\cfoot{}
\rfoot{}
\renewcommand{\headrulewidth}{0pt}
\renewcommand{\footrulewidth}{0pt}
% To make sure we actually have header 0.5in away from the top edge
%12pt is on-sixth of an inch. Subtract this from 0.5in to get headsep
%value
\setlength\headsep{0.333in}

\usepackage[normalem]{ulem}
\usepackage{amsmath}
\usepackage{textcomp}
\usepackage{amssymb}
\usepackage{capt-of}
\usepackage{hyperref}
\usepackage{setspace}
\doublespacing
\usepackage{enumitem}
\setlist[enumerate,itemize]{noitemsep,nolistsep,leftmargin=*}

\usepackage[american]{babel}
\usepackage{csquotes}
\usepackage[style=mla-new]{biblatex}
\addbibresource{writing-assignment-1-liberal-arts.bib}
%
% Works cited environment
% (to start, use \begin{workscited...}, each entry preceded by
% \bibent)
% - from Ryan Alcock's MLA style file
%
\newcommand{\bibent}{\noindent \hangindent 40pt}
\newenvironment{workscited}{\newpage \begin{center} Works Cited \end{center}}{\newpage }

%
% Begin document
%
\begin{document}
\begin{flushright}
\singlespacing
Matthew M. Nelson\\
Luke Matthews\\
Cultural Anthropology and Human Diversity\\
\today\\
\end{flushright}
\doublespacing

\begin{center}
  Writing Assignment \#1
\end{center}

\section*{Liberal Arts}

In the late \nth{20}-century, the term ``liberal'' accrued a largely negative
connotation in American society. This was in large part due to the term's
association with ``political correctness'' and the rise of identity politics.
While this association is somewhat warranted, the terms are not equivalent.
Since the Middle Ages, the ``liberal arts'' has referred to the studies
comprising the trivium and quadrivium, or grammar, logic, and rhetoric; and
arithmetic, geometry, music, and astronomy, respectively
\autocite[3]{Joseph.2002}. These subjects were distinguished from the
\emph{mechanical} or \emph{servile} skills employed in professional pursuits.

In his book, \uline{Thank You for Arguing}, Jay Heinrichs discusses the
importance of rhetoric to American society. He refers to a ``natural
aristocracy'' comprised of those with the best ``liberal education'' whom the
founders of the United States assumed would arise to ``refine and enlarge''
public opinion. The author defines ``liberal'' to mean free from dependence upon
others, and ``liberal arts'' to mean the skills that ``prepared students for
their place at the top of this merit system''
\autocite[366]{heinrichs2017thank}. This is an interesting distinction in our
society today which has devalued the \emph{mechanical} skills of traditionally
``blue collar'' trades, and overvalued higher education. It's also telling that
the founders of the U.S. depended upon an ``enlightened, disinterested few''
\autocite[367]{heinrichs2017thank} individuals beholden to neither factional or
economic interests. The idea being that these independently wealthy individuals,
without a proverbial ``horse in the race'', would shape public opinion in such a
way that decisions could be made without disproportionately benefiting any group
in particular.

In class, liberal was defined as someone who did not wish to live under the rule
of a king or emperor \autocite{Matthews.2017}. This anti-imperialist meaning,
while evident in much of U.S. history before the second World War, is more
likely labeled anarchist in today's political climate. This sense of the word,
however, is consistent with that which Hendricks uses. Of course, there remain
few true kings or emperors. The idea of ``divine right'' has little sway in our
world today. However, as noted in class, the former authority of kings has been
replaced by the pervasive influence and psychological warfare employed by
corporations and advertisers.

Through all of these facets of the term runs the thread of ``independence'' in
one form or another. It's this thread that is particularly important in the
education I seek here at Madison College. Cultivating the skills necessary to
make informed decisions, to form sound opinions is of primary importance. This
requires an ability to discern the socio-cultural and ideological motivations
behind arguments, both in popular media and academic research. To put it another
way, the liberal arts are important to honing a skeptical mind, another unjustly
maligned term.

\section*{Natural \& Cultural Categories}

In order to successfully navigate the world around us, human beings developed
the capacity for symbolic thought. This form of cognition differs from the
\emph{indexical} thought possesed throughout the rest of the animal kingdom in that it
allows us to generalize, or abstract, our experience and apply it to new
experiences. Categories are a manifestation of this capacity for abstraction.

Webster's defines \emph{category} as ``one of the highest classes to which the
objects of knowledge or thought can be reduced, and by which they can be
arranged in a system'' \autocite{porter1913webster}. The differentiation is
often what comes to mind when we think about categories; what properties
distinguish one object from another. What is more important to the concept is
the ability to form a system. That is, the relations between the members of
different categories are what allow us to derive value from the categories. The
category tree is only valuable inasmuch as it enables action on our part. An
example being that I can produce fire using a tree and a type of rock. The fact
is that most any tree will do, but not any rock will. This is the power of
abstraction.

Of primary interest to those of us involved in anthropology are \emph{natural}
and \emph{cultural} categories. \emph{Natural categories} are classifications
which depend upon an object possessing observable, inherent properties. Examples
of which include bipedality, flammability, viscosity, etc. Membership in each of
these categories is independent of the existence of human beings and any
particular cultural context.

Alternately, membership in \emph{cultural categories} is entirely dependent upon
the specific cultural background of the observer. Cultural categories, in fact,
form continua rather than delineate impermeable boundaries. For each cultural
category there is an ideal or authentic instance by which the purity of all
other instances is judged. Examples of this are fruits, race, and Tolkienian
fantasy. In each of these examples, membership is independent of any empirical
property of the observed object.

\emph{Human} and \emph{person} are, respectively, natural and cultural
categories. Classification as human is dependent upon properties such as being a
mammal, bipedal, and having fully-opposable thumbs. Classification as a person,
however, is entirely dependent upon the respect afforded an object by a human
being. I say object as it is entirely possible for a non-human object to be
afforded the status of personhood while denying it to a human. The denial of
personhood to human beings is seen in propaganda during every war or conflict. A
human enemy presented as either an agent of pure evil or as entirely
animalistic. This is done in order to stir the emotions and make it possible for
``our boys'' to murder people they have never before encountered and have no
other motivation to kill. This tactic is regularly employed in political speech
as we're encouraged to view political rivals as physically defective or wanting
in some way \textemdash{heartless for example}.

On the other side of the Venn diagram are non-humans that are afforded the
respect commensurate with being a person. Common examples being pets like dogs
and cats, but even more abstractly, legal entities like corporations in
America enjoy the same rights as any citizen.

\newpage{}
\printbibliography
\end{document}

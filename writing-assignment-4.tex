\documentclass[12pt]{article}
\usepackage[utf8]{inputenc}
\usepackage[T1]{fontenc}
%
% Margin - 1 inch on all sides
%
\usepackage[letterpaper]{geometry}
\usepackage{times}
\geometry{top=1.0in, bottom=1.0in, left=1.0in, right=1.0in}

%
% Double-spacing
%
\usepackage{setspace}
\doublespacing

%
% Rotating tables (e.g. sideways when too long)
%
\usepackage{rotating}

\usepackage{nth}


\usepackage{graphicx}
\usepackage{grffile}
\usepackage{longtable}
\usepackage{wrapfig}

%
% Fancy-header package to modify header/page numbering (insert last name)
%
\usepackage{fancyhdr}
\pagestyle{fancy}
\lhead{}
\chead{}
\rhead{Nelson \thepage}
\lfoot{}
\cfoot{}
\rfoot{}
\renewcommand{\headrulewidth}{0pt}
\renewcommand{\footrulewidth}{0pt}
% To make sure we actually have header 0.5in away from the top edge
%12pt is on-sixth of an inch. Subtract this from 0.5in to get headsep
%value
\setlength\headsep{0.333in}

\usepackage[normalem]{ulem}
\usepackage{amsmath}
\usepackage{textcomp}
\usepackage{amssymb}
\usepackage{capt-of}
\usepackage{hyperref}
\usepackage{enumitem}
\setlist[enumerate,itemize]{noitemsep,nolistsep,leftmargin=*}

\usepackage[american]{babel}
\usepackage{csquotes}
\usepackage[style=mla-new]{biblatex}
\addbibresource{writing-assignment-1-liberal-arts.bib}
%
% Works cited environment
% (to start, use \begin{workscited...}, each entry preceded by
% \bibent)
% - from Ryan Alcock's MLA style file
%
\newcommand{\bibent}{\noindent \hangindent 40pt}
\newenvironment{workscited}{\newpage \begin{center} Works Cited \end{center}}{\newpage }

\hypersetup{
 pdfauthor={Matthew M. Nelson},
 pdftitle={Writing Assignment \#4},
 pdfkeywords={},
 pdfsubject={},
 pdfcreator={Emacs 25.3.1 (Org mode 9.1.2)},
 pdflang={English}}

%
% Begin document
%
\begin{document}
\begin{flushright}
\singlespacing
Matthew M. Nelson\\
Luke Matthews\\
Cultural Anthropology and Human Diversity\\
\today\\
\end{flushright}
\doublespacing

\begin{center}
  Writing Assignment \#4
\end{center}

\section*{The Pigs and Power Question Set}

Influence and authority form a continuum from which those who occupy positions
of leadership derive power. And just as no society is neither purely egalitarian
nor purely hierarchical, leadership is never rooted solely in influence or
authority. That said, these two sources of power are useful tools for
understanding how leadership operates in a society. Ongka's role as \emph{big man} in
Kawelka society is difficult to understand when viewed from the standpoint of
industrialized Western society. This is because leaders in our society derive
power from \emph{authority}, primarily. The Kawelka \emph{big man}, on the other hand, derives
his power from \emph{influence}.

Authority depends upon the threat of violence and coercion, be it implicit or
explicit, to affect an outcome. Violence is disturbing and at odds with social
cohesion between individual. Thus violence as a source of sustainable social
power is only possible in a society large enough that most interactions between
its members are anonymous. Without anonymity, violence becomes increasingly
difficult the less social distance between the purveyor and recipient of that
violence.

Influence, from the anthropological point of view, is an expression of social
relationships and is highly contingent both upon the social distance between
parties and the concerned parties' social skills. The knowledge of a person's
desires, proclivities, past actions, and disposition, coupled with the ability
to leverage this knowledge form a means by which a person might affect the
behaviors of others. Thus, the \emph{big man} must rely on his own knowledge of the
members of his society and his own ability to employ that knowledge in
persuading others to act as he wishes.

There is a direct relationship between the size of a society and its leaders'
level of dependence on either influence or authority. Humans have an upper limit
on the number of social relationships they are able to maintain. As influence
requires low levels of social distance to be effective, the larger a society
grows the less its leaders will be able to maintain the degree of social
closeness required. And, as a society grows larger, social distance between
members necessarily increases as people are less able to keep track of each
other. Leadership in this larger society must increasingly depend upon the tools
of authority in order to maintain social cohesion.

In the case of Ongka, his role as \emph{big man} depends upon the tools of influence in
order to fulfill his obligation in the form of the Moka. Kawelka society is
based in the concept of reciprocity, of which the Moka is an example writ large.
The high level of interdependence and lack of anonymity with Kawelka society
creates an environment hostile to the tools of authority. Thus Ongka must
coax, cajole, and negotiate with his fellow villagers in order to assemble his
big Moka.

\newpage{}
\section*{The Personhood and Consumption Question Set}
Social distance is the ``perceived or desired degree of remoteness between a
member of one social group and the members of another, as evidenced in the level
of intimacy between them'' (``Social distance''). In day-to-day life, this
distance reveals itself in the preponderance of our relationships mediated by
money. For example, in a society with little social distance, if I need a bushel
of apples in order to bake a pie for Thanksgiving I will likely go to an orchard
and negotiate with the farmer directly. The exchange of goods and services in
this case is dependent upon my relationship with the farmer, my knowledge of him
and his business, and my ability to use this leverage to my advantage. This
exchange would look different in an industrialized Western society. The most
notable differences would be the generalized price, and thus the presence of
money, and the relative lack of social interaction between myself and the
farmer. It is entirely reasonable to expect that this interaction would only
last a few seconds in an industrialized Western society.

Anonymity is behind these differences. Smaller societies have little anonymity
as it is easy (or at least, conceivable) that all members of the society know
and depend upon each other to some extent. Anonymity appears once a society
grows to a size such that most of its members cannot maintain personal
relationships with all other members. At this size, people can no longer rely
personal relationships to inform exchange, and the need for a generic medium of
exchange arises. The most common example of this is currency in the form of
money.

The introduction of money into a society has a depersonalizing effect, both upon
relationships and goods. When I was able to haggle with the apple farmer, I
could perhaps offer her a stack of pork chops in exchange for that bushel of
apples. The farmer, knowing that I always bake apple pies this year gives me
Braeburn apples instead of the Galas I normally eat. In this exchange, both
parties have exploited some knowledge of the other. In doing so, the goods
assume an aspect of the relationship. I thank the farmer for remembering the
right apples as I'm baking the pie, and the farmer recalls the time the pork
chop on her plate escaped the pen and caused that ruckus in town. Now consider
this exchange again, but this time mediated by money. Now the apples cost \$5.
There's no reason to associate the apples with a particular farmer, and the
farmer has little reason to remember to whom she's selling.

This depersonalizing, anonymizing effect extends to the production of goods and
services as well. As the social relationships once engendered by the act of
exchange erode, the means of exchange reduces the products of labor to little
more than their monetary cost. This deincentivizes the continuation of
production legacies, or occupations, that shaped generations and societies.
Western society is rife with allusions to occupations. We name everything from
sports teams to cars to families after occupations. At one time, occupation was
a significant contributor to a person's identity. This is no longer true in many
Western societies. As manufacturing jobs have dwindled and corporations
propagated, Westerners have begun to distance themselves from their occupations.

Consumption has largely replaced occupation as a primary source of
identity-making material in Western societies. As exchange and production have
lost their social significance, and even disappeared from the landscape,
Westerners have begun to signify identity through the products we buy, the media
we watch or read, the foods we eat, and the ideologies to which we adhere. Where
once a person might identify themselves as a miner or engineer, today they're
more likely to identify themselves as a Cubs fan or steampunk, the former
occupations relegated to the status of ``just work.''

\begin{workscited}
  \bibent
  ``Social distance.'' \emph{New Oxford American Dictionary}, 2016. Oxford University Press.
\end{workscited}

\end{document}
